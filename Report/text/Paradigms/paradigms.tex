\section{Paradigms}
\label{sec:paradigms}
In this chapter we will discuss the software entrepreneurship paradigms and outline how we use one.

For software entrepreneurship there are two paradigms: Analyse, Design, Enact (ADE) and Consider, Do, Adjust (CDA).
The ADE paradigm relies on causal logic with a focus on predicting the future to control it.
It is a useful paradigm when the future is knowable, the goal is clear and the environment is reasonably well structured.
The CDA Paradigm instead relies on effectual logic with a focus on working with the things that can be controlled.
It is a useful paradigm when the future is unknowable, the means are clear, and the environment is subject to human shaping.

For our project we have a clear goal (a collection of streaming services), but are not sure which means we have available (which kind of agreement can we make with streaming services?).
Furthermore the environment is well structured with several other subscription streaming services to compare with, and use the comparison to get an idea of the future.
Therefore we choose the ADE paradigm.

When outlining the business process we consider it five phases because we use the ADE paradigm.

\subsection{Mobilize}
In the mobilization phase we would assemble our team, which would consist of us as programmers and maybe some sales specialist. It might also be a good idea too contact some streaming services to test the preliminary business idea.

\subsection{Understand}
The primary action for us in the understanding phase would be to research the existing streaming services. Their pricing and assortment would give us a better idea of our potential market position and analysing their client features would give us an idea of expected features for our client.
If we are in contact with the streaming services at this point we can also use their expert knowledge on the area.
Since our intended product is very similar to existing products with an established market, we are unlikely to get much out of studying potential customers.
%maybe something about netflix as an earlier failure

\subsection{Design}
In this phase we would probably focus on designing various business ideas and propositions for the streaming services. We might also make a prototype of our client.

\subsection{Implement}
Here we will contact the streaming services and negotiate an agreement with them based on our business ideas. We would then implement the system based on the agreement.

\subsection{Manage}
The manage phase is mostly a maintenance phase for our business. Here we could very well switch over to an iterative or agile approach as we focus on keeping our product the best through continuous improvement.

\subsection{Financial Issues}
Our biggest financial issue is that if we don't get a good agreement with the existing streaming services we will have difficulties being price competitive. We can make a vendor service that simply provides the convenience of browsing the services you have already paid for, but it is difficult to get much profit out of it. Either we charge for using our service which would make the total cost inarguable worse, or we use advertisements which risks hurting the convenience we sell our product on.