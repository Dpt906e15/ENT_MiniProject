\section{Business Model Canvas}
\label{sec:business_canvas}
In this chapter we will discuss the business canvas model for our idea, as well as pattern considerations and our design approach.

By building up a business model canvas as a concept for our project, we gain insight for discussion, and a common understanding for what we are doing. Using Osterwalder's approach \cite[p. 1-51]{canvas}, this is done through nine building blocks. In \figref{fig:model_canvas}, a graphical overview of the nine blocks can be seen. Each will be described with relation to the project throughout this chapter.

\begin{figure}[h]
    \begin{center}
        \includegraphics[scale=0.8]{./pics/model_canvas}
        \caption{The Business Model Canvas for our project}
        \label{fig:model_canvas}
    \end{center}
\end{figure}


\subsubsection*{Customer Segments}
Our product aims for an already existing market, although we want to change this market. Nonetheless, it is a mass market, consisting of people interested in several existing media streaming services.

\subsubsection*{Value Propositions}
We are focusing on the customers, who seeks several streaming services, collected in one place and at a lower price. That means there are three services of value for the customers: Convenience, accessibility and price. Convenience comes in collecting the product in one place, and making the interface painless to use. Accessibility is in the amount of suppliers, as more suppliers give wider range of streaming possibilities. Pricing lies in how far down we can push the price, which is also dependent on the suppliers, among other possibilities, such as advertisement.

\subsubsection*{Channels}
There are several ways of reaching our customers. The one way we are sure of is the fact that we want to offer our services through the same experience as existing services offer them: Streaming over the internet. It is important to note that this has to be done over several platforms, to gain competitive value through accessibility. This includes applications for mobile devices and for consoles. Advertisement of our product can be done in several ways, but one of the least costly ways to go about this is through the suppliers. As the customers are those already using the services that our suppliers offer, the best place to advertise is through these.

\subsubsection*{Customer Relationships}
As the existing companies do, being the suppliers of our product, we want to offer an automated service through a personal profile. Through this, it is possible to offer service without greater work effort, as well as the customer being able to mend their profile to their needs. As an extra feature, we want the customer to be able to tailor their own profile to their needs. They should only have access to and pay for the services needed, in the case they only want a subset of supplier services. With personal profiles, it is possible through algorithms to use knowledge on past activity to aid the user in their experience and to suggest future actions. If a customer seems to never use the services from a specific supplier, then we can give awareness of this, such that the user can tailor the profile further.

\subsubsection*{Revenue Streams}
Through the personal profile as an automated service, we want to create earnings through subscription fees. As the most successful streaming services have their income through this method, we see no reason to change this. We want to tempt customers to use our service by being cheaper than using all the services of the customers separately. Furthermore, to tempt customers to buy more, they are given greater offers the more suppliers they add to their profile.

\subsubsection*{Key Resources}
The most important resources for this project is partnerships with the suppliers and technical expertise. If we do not have suppliers, we don't have a service. If we do not keep the suppliers satisfied with our partnership, we lose them again. Technical expertise ensures a good, reliable client for the customers. The most needed resources are therefore intelligent and human resources.

\subsubsection*{Key Activities}
As the business comes up and running, the main activity is maintenance in both relationships and in customer support. If we do not continue our partnership with the suppliers, we cannot guarantee our main contribution from the service, and therefore lose customers. If the customers are unhappy with the product, we will lose them if we aren't able to locate and handle the problem.

\subsubsection*{Key Partnerships}
The media streaming distributors are our main suppliers, as they give access to the motion pictures our customers want to see. without these, the rest of the business will crumble. It is worth to note that this creates a great risk, as if we cannot live up to the expectations of our suppliers, we can lose these.

\subsubsection*{Cost Structure}
There are three major cost factors for this project: Hosting, wages and licenses. We need hardware, software locales and technical expertise for hosting our client. As we need expertise in some areas, such as marketing, there is a greater cost in waging. Furthermore, there is the licensing cost, as we must pay the suppliers for their service to us. We want to handle this as a percentile split, depending on the individual customer. A certain amount from each subscription is dedicated to the suppliers. The amount is based on how the customer has tailored the profile, and how much this individual is paying. Depending on what supplier services the customer has been using, the amount of money is then split, to give those that contribute to the individual customer a greater share of the profit. Another way is to make a fixed arrangement with each supplier, giving the possibility of a greater content variety, and therefore a greater attraction value for new customers.

\subsection{Pattern Considerations}
There generally exists five categories of business model patterns\cite[p. 52-119]{canvas}:
\begin{itemize}
\item Un-Bundling Business Models
\item The Long Tail
\item Multi-Sided Platforms
\item FREE
\item Open Business Models
\end{itemize}

The pattern to consider regarding our business idea is mostly the The Long Tail pattern where elements from the Multi-Sided Platform patterns can be used. Parts from the Open Business Models are also worth considering, as a lot of inspiration can be gotten from already existing services (HBO Nordic, Netflix, etc.), but that is strictly speaking not necessarily \emph{collaborating with outside partners}\cite[p. 109]{canvas}, so the consideration is on the other two.

The most fitting pattern is the Long Tail, as we want to provide both ``non-hit'' and ``hit'' products to all of our customers. It makes sense as our business idea fits mostly to the points presented in \emph{Business Model Generation}\cite[p. 75]{canvas}. Another hint that this pattern is the best fit for the project is the fact that our service is similar to Netflix and they used the Long Tail pattern. At least they used to, it seems like they are moving away from ``non-hit'' products and producing their own ``hit'' TV shows (House of Cards, Orange Is the New Black, etc.). The elements from the Multi-Sided Platforms that can be used are regarding the revenue stream, as we can possibly do a similar deal as Spotify and Telia. Telia provides with their mobile plan a Spotify subscription. If we can get a similar deal, the revenue loss can be subsidized by customers of our own service, but we will possibly reach a larger market.
% 1: Business model generation, 52-119
% 2: Business model generation, 109
% 3: Business model generation, 75
\subsubsection{Design Approach}
We have chosen to use the Customer Insight\cite{1} technique in our design approach. If we had the funds available, we could hire social scientists which could sketch profiles of the customer segment. As this is not the case, we have made use of the Empathy Map, which is also known as the ``really simple customer profiler''\cite{2}.

The first thing to do when using the empathy map, is to brainstorm all possible customer segments that one might want to serve with the business model. Then three promising candidates should be chosen, from which a single candidate is selected for the first profiling exercise. Then a customer from that customer segment is thought up, by giving the customer characteristics, such as a name, income, occupation, and so on. Afterwards a profile is build for the customer by asking and answering six questions.
These questions should be put on a white board or flip chart, as seen in figure \ref{fig:empathy_map}(Hence, empathy map), and the answers could be written on stick-it notes and placed on the questions.

The six questions are\cite{2}:
\begin{itemize}
\item What does the customer see? Describe what the customer sees in their environment
\item What does the customer hear? Describe how the environment influences the customer
\item What does the customer really feel and think? Try to sketch out what goes on in their mind
\item What does the customer say and do? Imagine what the customer might say, or how the customer might behave in public
\item What is the customer's pain? E.g. what are their biggest frustrations?
\item What does the customer gain? E.g. what does the customer truly want or need to achieve?
\end{itemize}

\begin{figure}[h]
    \begin{center}
        \includegraphics[scale=0.52]{./pics/empathy_map}
        \label{fig:empathy_map}
        \caption{Adapted empathy map from XPLANE\cite{3}}
    \end{center}
\end{figure}

The three candidate costumer segments we have come up with are:
\begin{enumerate}
\item Couples in the age range of 25-35, with children, that is ``forced'' to have cable because the children need their Saturday morning cartoons.
\item Students in the age range of 18-24, not living at home, limited income e.g. support from the government. Illegally downloads movies and TV-shows, as they can't afford cable or to buy them.
\item People in the age range of 16-35, illegally downloads movies and TV-shows because of ease-of-access.
\end{enumerate}

We are basing our customer on candidate 2 from the list. He is called John, he has no income except for support from the government, he is studying computer science at a university, he is single.

Question 1, what does he see?
\begin{itemize}
\item He lives in a small, one bedroom, apartment with no space for a TV
\item He is mostly surrounded by friends, as his family lives a couple of hours away, but will come visit occasionally 
\item His friends are people with similar interests as him self or study buddies, such as computers, video games, books and comics, TV shows and movies
\item He is not exposed to many offers in his daily routine, as he gets no advertisements from not owning a TV and he uses an ad blocker when surfing the internet
\end{itemize}

Question 2, what does he hear?
\begin{itemize}
\item His friends talk mostly about their shared interests
\item He is influenced by his friends and the internet, e.g. forums
\item He is not influenced by a lot of media channels, as he do not watch advertisements, and he do not visit social media often because people fills his wall with random games.
\item The most influential media channel is probably YouTube, where he gets the news he is interested in
\end{itemize}

Question 3, what does he really think and feel?
\begin{itemize}
\item \todo{I need help with what to put here, business model generation 131:Henrik}
\item He really want to be a video game developer
\end{itemize}

Question 4, what does he say and do?
\begin{itemize}
\item He has a friendly attitude in public, e.g. holding doors for people if necessary
\item He will mostly discuss ideas, talk about hobbies and so on
\item He enjoys a good discussion, does not have to be about anything, and he indulge in a little bit of gossiping 
\end{itemize}

Question 5, what is his pain?
\begin{itemize}
\item One of his biggest frustrations are having a hard time understanding a study related subject
\item An obstacle he faces, is that when he needs a break from studying, he watches an episode from a TV show, but keep watching more episodes because it is easier than studying
\end{itemize}

Question 6, what do he gain?
\begin{itemize}
\item \todo{I need help with what to put here, business model generation 131:Henrik}
\end{itemize}
% 1: Business model generation, 127-133
% 2: Business model generation, 131
% 3: Business model generation, 130