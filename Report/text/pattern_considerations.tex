\subsection{Pattern Considerations}
There generally exists five categories of business model patterns\cite{1}:
\begin{itemize}
\item Un-Bundling Business Models
\item The Long Tail
\item Multi-Sided Platforms
\item FREE
\item Open Business Models
\end{itemize}

The pattern to consider regarding our business idea is mostly the The Long Tail pattern where elements from the Multi-Sided Platform patterns can be used. Parts from the Open Business Models is also worth considering, as a lot of inspiration can be gotten from already existing services (HBO Nordic, Netflix, etc.), but that is strictly speaking not necessarily \emph{collaborating with outside partners}\cite{2}, so the consideration is on the other two.

The most fitting pattern is the Long Tail, as we want to provide both ``non-hit'' and ``hit'' products to all of our customers. All of the points presented here \cite{3}. Another hint that this is the correct one, is that our service is similar to Netflix and they used the Long Tail pattern. At least they used to, it seems like they are moving away from ``non-hit'' products and producing their own ``hit'' TV shows (House of Cards, Orange Is the New Black, etc.). The elements from the Multi-Sided Platforms that can be used is regarding the revenue stream, as we can possibly do a similar deal as Spotify and Telia. Telia provides with their mobile plan a Spotify subscription. If we can get a similar deal, the revenue loss can be subsidized by customers of our own service, but we will possibly reach a larger market.
% 1: Business model generation, 52-119
% 2: Business model generation, 109
% 3: Business model generation, 75