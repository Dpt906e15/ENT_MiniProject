\subsection{Resource-based theory}

Generally, a company could be perceived as a collection of resources and certain capabilities which defines the resource-based theory of the company.
When these resources are combined in the right way, they form organizational capabilities.

In our case, we try to combine technically talented people (software developers and engineers,system architects,UI and integration specialists)and the appropriate software tools with the collaboration of our competitors (through the use of the media patents they have on specific tv series, movies etc.) in order to create a competitive advantage. This competitive advantage is described by the idea that our platform will provide access to media from both our biggest competitors (Netflix and HBO) in one place with the use of only one subscription and the option to pick specific shows from both.

The key resources we consider for our company are:

\begin{enumerate}
  \item Brand
  \item Software Products
  	\begin{enumerate}
   		\item Platform
   		\item Algorithms  	
    	\item Protocols
    	\item Embedded software
    	\item Associated hardware
  	\end{enumerate}
  \item People/Talent
  	\begin{enumerate}
      	\item Programmers
      	\item Project Managers
      	\item System Architects
      	\item Embedded software
    \end{enumerate}
  \item Core Competencies
  	\begin{enumerate}
        	\item Software processes and methods
        	\item Knowledge
        	\item Community
    \end{enumerate}
  \item Vision Direction
  \item Strategies
  \item Market Orientation
  \item Positioning and Targeting
\end{enumerate}

These resources can be added to certain capabilities of the company, which in turn will define its competitive advantage. The defining capabilities in our case are:

\begin{enumerate}
  \item Learn new programming skills
  \item Deliver quality through our platform and services
  \item Respond to feedback and customer needs
  \item Expand existing features and develop new ones
\end{enumerate}