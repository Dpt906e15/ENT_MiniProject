\subsection{Resource-based theory}

Generally, a company could be perceived as a collection of resources and certain capabilities which defines the resource-based theory of the company \cite[p. 13-14]{fiveForces}.
When these resources are combined in the right way, they form organizational capabilities.

In our case, we try to combine technically talented people (software developers and engineers,system architects,UI and integration specialists)and the appropriate software tools with the collaboration of our competitors (through the use of the media patents they have on specific tv series, movies etc.) in order to create a competitive advantage. This competitive advantage is described by the idea that our platform will provide access to media from both our biggest competitors (Netflix and HBO) in one place with the use of only one subscription and the option to pick specific shows from both.

The key resources we consider for our company are:

\begin{enumerate}
  \item \textbf{Brand}\\
In business, branding plays a pivotal role in defining a company. A company's brand is what differentiates and distinguishes it from its competitors and what makes it stand out across as unique and established. 
A company communicates with its customers through its brand, a promise which relies the information of what products and services they can expect. Microsoft, Apple, Samsung and many other companies are easily recognized through their logos and the products and services they provide. In our case, we build the company around our media unifying platform called "Watchr" where both the company and the platform have distinguishable and unique logos.
\pagebreak

  \item \textbf{Software Products}
  
  	\begin{enumerate}
   		\item \textbf{Platform}\\
Currently, our business plan revolves around Watchr, our media platform and its success is essential for the success of the company as a whole.
	    \item \textbf{Algorithms and protocols}\\ 
The underlying structure of Watchr will build upon proprietary algorithms and specific protocols which will be used for establishing the right user experience for our customers. An example of such could be a recommendation system adapting based on users' preferences, sorting and filtering algorithms for different criteria. 
    	\item \textbf{Hardware}\\
Similarly to how our competitor's media streaming services work, we would have dedicated physical servers for storing all the content on Watchr along with all the user-related information. 
  	\end{enumerate}
  \item \textbf{People/Talent}\\
People are the most important resource for a company in order to function correctly and be successful. Since we are building a media platform, we need people with technical expertise in programming, system design,architecture and integration. Additionally, we would need people with the right aptitude to maintain the system and project managers to guide and manage everyone else with further development and new features.
  \item \textbf{Core Competencies}
  	\begin{enumerate}
        	\item \textbf{Software processes and methods}\\
In order to implement Watchr, the technical people in the company should adhere to a software development methodology and use specific software processes to facilitate that. A good choice would an iterative approach such as SCRUM.
        	\item \textbf{Knowledge}\\
The knowledge a company has access to is another valuable commodity. In our case this can include things ranging from technical expertise, business knowledge about the market, to social relations with partners and other companies.
        	\item \textbf{Community}\\
A company which has direct access to its community, or namely its customers, can greatly benefit from its feedback on certain features of the product or service at hand, as well as how some parts could be improved and revised. Therefore, a good idea would be to implement a direct way for our customers to contact us through Watchr.
    \end{enumerate}
  \item \textbf{Vision Direction}\\
The ongoing vision of the company is to unite all the streaming media available on other separate platforms in one place which will facilitate the access for everyone to his/hers favourite tv shows, series etc.
\end{enumerate}

These resources can be added to certain capabilities of the company, which in turn will define its competitive advantage. The defining capabilities in our case are:

\begin{enumerate}
  \item Learn new programming skills
  \item Deliver quality through our platform and services
  \item Respond to feedback and customer needs
  \item Expand existing features and develop new ones
\end{enumerate}