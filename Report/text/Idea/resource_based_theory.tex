\subsection{Resource-based theory}

Generally, a company could be perceived as a collection of resources and certain capabilities which defines the resource-based theory of the company.
When these resources are combined in the right way, they form organizational capabilities.

In our case, we try to combine technically talented people and the appropriate software tools with the collaboration with our competitors in order to create a competitive advantage.

The key resources we consider for our company are:

\begin{enumerate}
  \item Brand
  \item Software Products
  	\begin{enumerate}
   		\item Platform
   		\item Algorithms  	
    	\item Protocols
    	\item Embedded software
    	\item Associated hardware
  	\end{enumerate}
  \item People/Talent
  	\begin{enumerate}
      	\item Programmers
      	\item Project Managers
      	\item System Architects
      	\item Embedded software
    \end{enumerate}
  \item Core Competencies
  	\begin{enumerate}
        	\item Software processes and methods
        	\item Knowledge
        	\item Community
    \end{enumerate}
  \item Vision Direction
  \item Strategies
  \item Market Orientation
  \item Positioning and Targeting
\end{enumerate}

\todo{(Svetomir)Should add capabilities list}

These resources can be added to certain capabilities of the company, which in turn will define its competitive advantage.