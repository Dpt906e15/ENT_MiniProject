\section{Paradigm}
For software entrepreneurship there are two paradigms: Analyse, Design, Enact (ADE) and Consider, Do, Adjust (CDA).
The ADE paradigm relies on causal logic with a focus on predicting the future to control it.
It is a useful paradigm when the future is knowable, the goal is clear and the environment is reasonably well structured.
The CDA Paradigm instead relies on effectual logic with a focus on working with the things that can be controlled.
It is a useful paradigm when the future is unknowable, the means are clear, and the environment is subject to human shaping.

For our project we have a clear goal (a collection of streaming services), but are not sure which means we have available (which kind of agreement can we make with streaming services?).
Furthermore the environment is well structured with several other subscription streaming services to compare with, and use the comparison to get an idea of the future.
Therefore we choose the ADE paradigm.

the phases or activities

money troubles